\documentclass[11pt, a4paper]{article}

% One of the following is required: problemset, recitation, quiz, exam
% The following are required: handoutnum, assigneddate.
% If there is only one date, set both duedate and assigneddate to be the same.
% Do not change handoutnum or dates
\usepackage[
problemset,
handoutnum=3,
assigneddate={5 October 2021},duedate={15 October 2021},
% % Uncomment the line below IF these are solutions
solution,
% % Uncomment the line below IF you are a student submitting solutions
student,
% % Replace with your name
name={Lei Zhang},
% Replace with names of all group members who collarborated on this.
% If unsure about ordering, then you can follow the convention in theory
% and list names alphabetically by last name.
groupmembers={}
]{course-handouts-preamble}

% Be careful of commas and put text with spaces within {curly braces}
% Don't use a comma at the end, but do use commas between options.
% Weird errors occur otherwise, I wasted some time failing to debug those.
% DO NOT EDIT
% These are fixed values that should not be changed during this course.
\pgfkeys{/chp/fixed/.cd,
instructorname = {Ravi Sundaram},
coursename = {CS5800 Algorithms}}

% Add any macros you want below, or put them in a separate file and \input{file}
% keeping the preamble clean can keep you sane.

\begin{document}
% Do not change either of the below lines.
\insertHandoutInfoBox{}
\ifbool{isexam}{\input{exam-blurb}} %comment this line only if it throws an error.

% Start adding content from below here.


\newproblem{Greedy Coins}{10+5}
\begin{enumerate}
    \item Recall the change-making algorithm from lecture. Suppose that the available coins are in the denominations that are powers of $c$, i.e., $c^0, c^1,\dots,c^k$ for some integers $k > 0$, $c > 1$. Prove that the greedy algorithm always yields an optimal solution.
    \item Give a set of coin denominations for which the greedy algorithm does not give an optimal solution. Your set should include a penny so there's a solution for every change amount $n$.
\end{enumerate}

\begin{solution}
    \begin{enumerate}
        \item Proof: The optimal solution must follow the rule that the number of the coins in the denominations of $c^i$ where $0 \leq i < k$ has to be less than $c$. Because any $c$ coins in the denomination of $c^i$ where $0 \leq i < k$ can be replaced by one coin of $c^{i+1}$, thus indicating the solution is not optimal. As the greedy algorithm ensures that there won't be $c$ identical coins of value between $c^0$ and $c^{k-1}$, it always yields the optimal solution.
        \item Coinage: $\{1, 10, 25\}$. \\
              For value of $30$, the greedy algorithm gives the solution of $[25,1,1,1,1,1]$, but the optimal solution is $[10,10,10]$.
    \end{enumerate}
\end{solution}


\newproblem{Super or normal}{25}
You have one supercomputer and $n$ normal computers on which you need to run $n$ jobs. Each job $i$ first spends $s_i$ time on the supercomputer and then $n_i$ time on the normal computer. A job can only start running on any of the normal computers after it has finished on the supercomputer. However, as soon as any job finishes on the supercomputer, it can immediately start on one of the free normal computers.
The goal is to finish running all the jobs as soon as possible. Note that since there is only one supercomputer, you'll always have to wait $\sum_{i=1}^{i=n} s_i$ time so that the jobs finish running on the supercomputer. However, you can optimize when you run the jobs on the normal computers to try to finish running all the jobs as soon as possible.
Show the following schedule is optimal: execute jobs after sorting them in \textit{decreasing order} by $n_i$.

\begin{solution}
    The idea is to determine which job should be run last. As the last job always has to wait for $\sum_{i=1}^{i=n} s_i$ time to run on a normal computer, the optimal should run the job with minimum $n_i$ last. Similarly, jobs with smaller $n_i$ will run later.\\

    Proof by induction: \\
    \textbf{Base case:} The job with minimum $n_i$ will be the last to run.\\
    \textbf{Inductive step:} Assume that the last $k$ jobs to run are the $k$ jobs with minimum $n_i$ time, i.e., the last $k$ terms from $(n-k+1)$ to $n$ when sorted in descending order by $n_i$.\\
    Let $S_k$ be the sum of supercomputer time of the last $k$ jobs. Then, the in no matter what order, the $(n-k)$th job has to wait for $\sum_{i=1}^{i=n} s_i - S_k$ time to run on a normal computer. Thus, it should be the job with the least $n_i$ time, i.e., the $(n-k)$th term of the list where jobs are sorted in descending order by $n_i$.\\
    Therefore, the last $k+1$ jobs to run are the jobs with the minimum $n_i$ time.\\
    \textbf{Q.E.D.}
\end{solution}

\newproblem{Distinct Edge Weights and MSTs}{10+10+10}
Assume that all the weights in the undirected graph $G$ are unique (no two edge weights are the same). Prove the following, or provide a counter example.
\begin{enumerate}[label=\alph*.]
    \item The MST is unique.
    \item The edge with the smallest weight $e_{min}$ always belongs to the MST.
    \item The edge with the largest weight $e_{max}$ never belongs to the MST.
\end{enumerate}

\begin{solution}
    \begin{enumerate}
        \item True. \\
              Proof: Suppose that there are two MSTs $T_1$ and $T_2$, and let $e$ be the lightest edge that exists in only one of the MSTs, say $e$ is in $T_1$. Then add $e$ to $T_2$ will create a cycle and there must be another edge $f$ on the cycle that only exists in $T_2$ and that $e$ is lighter than $f$ since $e$ is the minimum edge that exists in one of the MSTs. Thus, $T^* = T_2 \cup \{e\} - \{f\}$ is also a MST and that $w(T^*) < w(T_2)$, which contradicts that $T_2$ is a MST.
        \item True. \\
              Proof: Suppose that the lightest edge $e$ does not belong to the MST $T$. Then add $e$ to $T$ will create a cycle and there must be another edge $f$ on the cycle that is heavier than $e$. And $T^* = T \cup \{e\} - \{f\}$ is also a MST and that $w(T^*) < w(T)$, which contradicts that $T$ is a MST.
        \item False.\\
              Proof: Consider a linear graph where each edge is on the same line. Then MST for this graph must contain the heaviest edge.
    \end{enumerate}
\end{solution}

\newproblem{Programming: An Escape from time}{30}
Ulrich from 2019 travels to Winden 1953 to accomplish a secret mission. He travels using the popular Winden Caves. After he has finished his secret mission in Winden - 1953, he now has to return to the 2019. He can travel to 2019 using the caves. However, the cops of Winden know that Ulrich is on the loose, and are attempting to stop him by spreading his picture to the police at their bus stop.
You must help Ulrich escape Winden-1953 by determining which busses he should take to reach the caves. However, because the police are actively looking for him, Ulrich has decided that he cannot sit still at any one bus stop for too long. Thus you need to determine which busses he should take so that his layover is never too long.
\noindent
\textbf{Input Format}:
Line $1: ~N~ M~ T~$
\begin{itemize}
    \item [--]$N$ is the max label of any bus station (1 is the starting
          point, $N$ is the destination  outside the country). Ulrich always
          starts at station 1 at time 0.
    \item [--]$M$ is the number of bus rides which appear on the schedule.
    \item [--]$T$ is the max time Ulrich can spend at any one station without
          being  caught by  the  Winden Police.  Note  that he  is
          allowed to  spend exactly  $T$ time  steps at a  station, but  not a
          moment more.
\end{itemize}
Next $M$ Lines: $x$ $y$ $t_1$ $t_2$.
\begin{itemize}
    \item [--]$x$ is the station label where a bus will be leaving from
    \item [--]$y$ is the station label where this bus will arrive
    \item [--]$t_1$ is the time the bus will leave station $x$.
    \item [--]$t_2$ is the time the bus will arrive at station $y$.
\end{itemize}
\noindent
\textbf{Output Format}:
``NO'' if, no matter what busses Ulrich takes, he will be caught before reaching station $N$.
``YES $T_{min}$'' if he can escape , where $T_{min}$ is the earliest possible time that he arrives at station $N$.

At the very start, Ulrich is at station 1, at time 0. Suppose that Ulrich arrives at station $x$ at time $t$.
He can move to station $y$ if and only if:
\begin{itemize}
    \item [--] There is a bus going from $x$ to $y$ at time $t_1$, where $t_1 > t$, but $t_1 - t \leq T$, the max time he can spend at the station.
    \item [--] If there is such a bus ride, labeled as $x,~ y,~ t_1,~ t_2$, then after taking this bus, he will be at station $y$ at time $t_2$. Furthermore, he will have spent $t_1 - t$ time units waiting at station $x$.
\end{itemize}

\textbf{Sample Input}:
4 5 20
1 2 15 20
1 3 1 6
2 4 25 30
3 4 2 7
3 4 30 35

\textbf{Sample Output}:
YES 30

\textbf{Explanation}:
Ulrich could take the following busses:
* Stop 1 to Stop 2, leaves at 15, arrives at 20
* Stop 2 to Stop 4, leaves at 25, arrives at 30

The only other bus that arrives earlier at stop 4 leaves stop 3 at time 2, but it is not possible to get to stop 3 before time 2.

\begin{solution}
    \begin{enumerate}
        \item HankerRank username: \textbf{zhang\_lei1}
        \item Submission handle: \url{https://www.hackerrank.com/contests/cs5800-fall21-pset-3/challenges/cs5800-spy-escape/submissions/code/1337704734}
    \end{enumerate}
\end{solution}

\end{document}